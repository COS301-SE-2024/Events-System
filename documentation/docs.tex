\documentclass[a4paper,12pt]{article}
\usepackage{graphicx}
\usepackage{amsmath}
\usepackage{hyperref}
\usepackage{enumitem}
\usepackage{float}

\title{Events System Documentation}
\author{Bieber Fever}
\date{\today}

\begin{document}

\maketitle
\tableofcontents

\section{Introduction}
\label{sec:introduction}

\section{User Stories / User Characteristics}
\label{sec:user-stories}

\subsection{User Characteristics}

\subsubsection{Host}
Hosts are responsible for organizing, scheduling, and managing events within the organization. 
They create event listings, provide detailed descriptions, set dates, and handle logistics to 
ensure successful execution. Their primary goal is to promote employee bonding, engagement, and 
participation by curating events that are relevant and interesting to employees. They aim to foster 
a sense of community and enhance the workplace culture. Hosts are typically detail-oriented, possess 
strong organizational skills, and have a good understanding of the interests and needs of the employees. 
They should be adept at communication and marketing to effectively promote events. Hosts will require 
tools to manage event details, send invitations, track RSVPs, and gather feedback post-event. They 
will also need access to analytics to assess event success and employee engagement levels. 

\subsubsection{Employee}
Employees are the end-users of the Event System. They browse the events, RSVP to those they are 
interested in, and participate in the activities organized by the Hosts. They aim to find events 
that align with their interests, professional growth, or social engagement needs. Participation 
in these events is expected to enhance their sense of belonging and satisfaction within the organization. 
Employees vary widely in their interests and engagement levels. The system needs to be user-friendly to accommodate 
all employees, regardless of their technical proficiency. Employees will require a calendar view of upcoming events, 
search and filter options to find relevant events, RSVP capabilities, and reminders for events they plan to attend. 
They also need the option to provide feedback on events they attend.

\paragraph{Event Creation}

\begin{enumerate}
    \item As a Host, I want to create new events in the system so that employees can view and RSVP to them.
    \item As a Host, I want to edit event details so that I can update information if there are any changes.
    \item As a Host, I want to delete events that are no longer relevant so that the event list remains current.
\end{enumerate}

\paragraph{Event Details}

\begin{enumerate}
    \item As a Host, I want to add multimedia elements like images and videos to the event details so that the event 
    is more engaging and informative.
    \item As an Employee, I want to view detailed information about an event, including time, date, location, host, 
    participants, description, and preparation details, so that I can decide whether to attend.
    \item As an Employee, I want to see a map or directions to the event location so that I can easily find my way there.
\end{enumerate}

\paragraph{Calendar View}

\begin{enumerate}
    \item As a Host, I want to see a calendar view of all my events so that I can manage them efficiently.
    \item As an Employee, I want to see a calendar view of upcoming events so that I can easily plan my schedule.
    \item As an Employee, I want to add events to my personal calendar so that I can keep track of my commitments.
\end{enumerate}

\paragraph{Filtering}

\begin{enumerate}
    \item As an Employee, I want to filter events based on geolocation so that I can find events happening near me.
    \item As an Employee, I want to filter events based on social club so that I can find events related to my interests.
    \item As an Employee, I want to filter events using other criteria like date and category so that I can quickly 
    find relevant events.
    \item As an Employee, I want to filter events based on the type (e.g., in-person, virtual) so that I can find 
    events that suit my preference.
\end{enumerate}

\paragraph{Search}

\begin{enumerate}
    \item As a Host, I want to search for past events to review feedback and attendance data so that I can improve future events.
    \item As an Employee, I want to search for events using keywords so that I can find specific events quickly.
\end{enumerate}

\paragraph{RSVP}

\begin{enumerate}
    \item As a Host, I want to view the list of RSVPs for my events so that I can prepare accordingly.
    \item As an Employee, I want to RSVP to events so that the host knows I plan to attend.
    \item As an Employee, I want to change my RSVP status if my plans change so that the host has an accurate count of attendees.
\end{enumerate}

\paragraph{Series Subscription}

\begin{enumerate}
    \item As an Employee, I want to subscribe to a series of events so that I receive notifications for each event in the series.
    \item As an Employee, I want to manage my subscriptions to event series so that I can opt in or out as needed.
\end{enumerate}

\paragraph{Historical Events}

\begin{enumerate}
    \item As an Employee, I want to view a list of historical events I have attended so that I can keep track of my past engagements.
    \item As an Employee, I want to access materials or recordings from past events so that I can revisit the content.
\end{enumerate}

\paragraph{Personal Information}

\begin{enumerate}
    \item As a Host, I want to view and edit my profile information so that my contact details and preferences are up-to-date.
    \item As an Employee, I want to set my personal information, such as dietary requirements, in the system so that 
    I do not need to specify it each time I RSVP.
    \item As an Employee, I want to update my personal information easily so that my preferences are always current.
\end{enumerate}

\paragraph{Notifications}

\begin{enumerate}
    \item As a Host, I want to send notifications to employees about event updates or reminders to ensure maximum participation.
    \item As an Employee, I want to receive notifications about upcoming events so that I do not miss them.
\end{enumerate}

\paragraph{User Management}

\begin{enumerate}
    \item As a Host, I want to register an account on the system so that I can create events.
    \item As a Host, I want to sign in and verify my credentials so that I can securely access my account.
    \item As an Employee, I want to register an account on the system so that I can access event details and RSVP.
    \item As an Employee, I want to sign in and verify my credentials so that I can securely access my account.
\end{enumerate}

\paragraph{Analytics}

\begin{enumerate}
    \item As a Host, I want to view analytics and insights into employee participation in events so that 
    I can understand engagement levels and improve future events.
\end{enumerate}

\paragraph{Event Registration with QR Code}

\begin{enumerate}
    \item As a Host, I want to provide a scannable QR code for event registration so that attendance 
    tracking is streamlined and efficient.
    \item As an Employee, I want to easily indicate my attendance at events by scanning a QR code on 
    a signboard so that my presence is recorded without manual check-ins.
\end{enumerate}

\section{Functional Requirements}
\label{sec:functional-requirements}

\section{Service Contracts}
\label{sec:service-contracts}

\section{Class Diagram}
\label{sec:class-diagram}
\label{sec:class-diagram}
\begin{figure}[H]
    \centering
    \includegraphics[width=\textwidth]{EventsClassDiagram.png}
    \caption{Class Diagram}
    \label{fig:class-diagram}
\end{figure}

\section{Architectural Requirements}
\label{sec:architectural-requirements}

\subsection{Quality Requirements}
\label{subsec:quality-requirements}

\subsection{Architectural Patterns}
\label{subsec:architectural-patterns}

\subsection{Design Patterns}
\label{subsec:design-patterns}

\subsection{Constraints}
\label{subsec:constraints}

\section{Technology Requirements}
\label{sec:technology-requirements}

\end{document}

